\documentclass[12pt,UTF8]{ctexbook}
\usepackage{ctex}
\usepackage{caption}
\usepackage{graphicx}
\usepackage{float}
\usepackage{wrapfig}
\usepackage{array}
\usepackage[table, dvipsnames, svgnames, x11names]{xcolor}
\usepackage{colortbl}% 
\usepackage{tabularx}
\usepackage{amsmath, tikz}
\usepackage{amssymb}
\usepackage{xfrac}
\usepackage{eucal}
\usepackage{titlesec}
\usepackage{amsthm}
\usepackage{tikz-cd}
\usepackage{enumitem}
\usepackage{verbatim}
\usepackage{fontspec,xunicode,xltxtra}
\usepackage{xeCJK} 

\definecolor{gl}{RGB}{246, 252, 240}
\definecolor{gd}{RGB}{236, 244, 230}
\definecolor{bg}{RGB}{242, 244, 228}


\setCJKmainfont[BoldFont=STZhongsong]{STSong}
\setCJKmonofont{simkai.ttf} % for \texttt
\setCJKsansfont{simfang.ttf} % for \textsf
\setlength\parskip{8pt}
\setlength{\fboxsep}{12pt}
\renewcommand\thesection{\arabic{chapter}.\arabic{section}}

\usetikzlibrary{calc,topaths}

\newcommand\widearc[1]{%
\tikz[baseline=(wideArcAnchor.base)]{
    \node[inner sep=0] (wideArcAnchor) {$#1$}; 
    \coordinate (wideArcAnchorA) at ($(wideArcAnchor.north west) + (0.15em,0.1em)$);
    \coordinate (wideArcAnchorB) at ($(wideArcAnchor.north east) + (0.0em,0.1em)$);
%
    \draw[line width=0.1ex,line cap=round,out=45,in=135] (wideArcAnchorA) to (wideArcAnchorB);
}}

\newtheorem{df}{定义}[section] 
\newtheorem{pp}{命题}[section]
\newtheorem{tm}{定理}[section]
\newtheorem{ex}{例子}[section]
\newtheorem{sk}{思考}[section]
\newtheorem{po}{公理}
\newtheorem*{so}{解答}
\newtheorem*{proof2}{证明}
\newtheorem{xt}{习题}[section]
\newtheorem{cor}{推论}[pp]
% 列举环境的行间距
\setenumerate[1]{itemsep=0pt,partopsep=0pt,parsep=0pt,topsep=0pt}
\setitemize[1]{itemsep=0pt,partopsep=0pt,parsep=0pt,topsep=0pt}
\setdescription{itemsep=0pt,partopsep=0pt,parsep=0pt,topsep=0pt}
\setlength{\intextsep}{2pt}%
\setlength{\columnsep}{2pt}%
% 新函数
\renewcommand\parallel{\mathrel{/\mskip-4mu/}}
% 章节字体大小
\titleformat{\section}{\zihao{-2}\bfseries}{ \thesection }{16pt}{}
% 封面
\title{\zihao{0} \bfseries 极简数学·中学篇 \\ 第五册}
\author{\zihao{2} \texttt{大青花鱼}}
% \date{\bfseries\today}
\date{}
% 正文
\begin{document}
\maketitle
\tableofcontents
\newpage

\chapter{圆}

学习反比例函数和二次函数时,我们发现,就算是简单代数式的函数,它的图像也是我们无法手动画出的曲线。
曲线是比直线更复杂的形状。为了给我们今后研究各种曲线打下基础,以下我们研究一种简单的曲线:圆。

\section{圆的基本性质}
我们已经学过圆的概念。公理体系中,我们这样定义圆:平面上到定点$O$距离为定长的点的集合,是一个圆。
给定线段$XY$,到$O$的距离和$AB$等长的点构成一个圆。$O$叫做\textbf{圆心},
$XY$叫做圆的\textbf{半径},长度一般记为$r$,不至于混淆的时候,半径的长也简称为半径。

圆心为$O$、半径为$r$的圆,一般记为圆$(O, r)$或$\odot(O, r)$。
圆心$O$和另一点$P$确定的圆,一般记为圆$(O, P)$或$\odot(O, P)$。如果不在意半径,不至于混淆的情况下,
也可以简记为圆$O$。

平面上的点到$O$的距离小于$r$,就说它在圆内;如果等于$r$,就说它在圆上;如果大于$r$,就说它在圆外。

关于圆,我们有以下公理:
\begin{itemize}
    \item 直线和圆有两个交点$A$、$B$,当且仅当直线有部分在圆内。
    \item 给定点$A$、$B$和线段$EF$、$GH$,如果$|EF| + |GH| > |AB| > \left||EF| - |GH|\right|$,
    那么总存在两点$P$、$Q$,使得$|AP| = |EF|$、$|PB| = |GH|$,$|AQ| = |EF|$、$|QB| = |GH|$。
    $P$、$Q$分别在直线$AB$两侧。
\end{itemize}
连接圆上两点$A$、$B$,直线$AB$和圆有$A$、$B$两个交点,根据圆公理,线段$AB$(除端点)在圆内。
我们把线段$AB$称为圆的一条\textbf{弦}。
连接圆上一点$A$和圆心$O$,延长$AO$,根据圆公理,它和圆有另一个交点$B$,称为点$A$的\textbf{对径点}。
$AB$称为圆的\textbf{直径}。直径是过圆心的弦。它的长度是半径的两倍。不至于混淆的时候,直径的长也简称为直径。

给定圆上两点$A$、$B$,考虑弦$AB$的垂直平分线$l$,圆心$O$显然在$l$上。也就是说,恰有一条直径垂直平分每条弦。

圆、角和旋转有天然的关系。给定角$AOB$,可以定义\textbf{旋转}:

\begin{df}\label{df:0-0-0}
    给定角$AOB$,平面中一点$P$关于$\angle AOB$旋转的结果,
    是唯一使得$\angle POQ = \angle AOB$且$|OP| = |OQ|$的点$Q$。
\end{df}
$O$称为旋转的\textbf{中心}。任何点$P$绕中心旋转,结果都在圆$(O,P)$上。

可以看到,给定一个圆$(O,P)$,从点$P$出发,旋转不同的角度,
就得到圆上其它的点。从零角出发,随着角度不断增大,直到周角,我们沿逆时针经历了圆上所有的点。
也就是说,我们认为零角到周角的角按角度和圆上的点之间有一一映射。
换句话说,数轴上$0$和$360$之间的数,和圆上的点之间有一一映射。
我们把它称作\textbf{圆映射},记为$\gamma_{(O,P)}$。

通过$\gamma_{(O,P)}$,我们可以把对圆的研究,改为对数轴上线段的研究。
这样就把曲线上的问题转为了直线上的问题。
比如,既然$[0, 360)$对应整个圆,那么$[0,180]$就对应半个圆,
$[0,60]$就对应六分之一个圆,等等。我们把闭区间对应的圆的部分称为\textbf{圆弧}。

同一圆上两个圆弧分别对应$[a_1, a_1+x]$和$[a_2, a_2+x]$,这两个圆弧有什么不同吗?
观察圆的图像可知,并没有不同。也就是说,圆弧的形状只和它对应数轴上区间的长度有关,和它所在的位置无关。
只要对应的区间一样长,那么圆弧就全等,可以相互覆盖。换句话说,圆弧只要等长,就是全等的。
于是,线段所满足的公理,对同一个圆上的圆弧也成立。

和线段一样,圆弧也有起点和终点。比如$[0,60]$对应的圆弧,起点就是$P$,
终点是$60$度角$POQ$的终边和圆的交点$Q$。如果圆弧对应的区间长度超过$180$,就说它是\textbf{优弧};
如果圆弧对应的区间长度小于$180$,就说它是\textbf{劣弧};如果等于$180$,就说它是\textbf{半圆}。
优弧比半圆长,劣弧比半圆短。

从直线和圆相交的角度来看,圆上两点表示这两点确定的直线将圆分为两个圆弧。这两个圆弧并起来就是圆,
所以要么一个是优弧、一个是劣弧,要么两者都是半圆(这时直线过圆心)。

同一个圆上,明确了起点$A$和终点$B$,就唯一确定了圆弧$\widearc{AB}$。如果只说了两点$A$、$B$,
那么$\widearc{AB}$一般指劣弧或起点为$A$终点为$B$的圆弧。如果要指优弧,一般会特别强调。

\begin{xt}\label{xt:0-0-0}
    证明:\\
    \indent 1. 同一个圆中,直径是最长的弦。\\
    \indent 2. 任意线段经过旋转得到等长的线段。任意三角形经过旋转得到同角全等的三角形。 
\end{xt}

\section{圆心角和圆周角}

根据圆映射的定义,每个圆弧都对应一个顶点在圆心,大小介于零角和周角之间的角,称为它的\textbf{圆心角}。
圆弧还可以对应另一类角。给定起点为$A$,终点为$B$的圆弧$\widearc{AB}$和圆上一点$P$,则角$APB$称为一个\textbf{圆周角}。
每个圆弧只对应一个圆心角,但可以对应很多个圆周角。

同一段圆弧的圆心角和圆周角之间,有什么关系呢?如右图,连接$PO$,延长交圆于对径点$Q$。
由于$\triangle AOP$是等腰三角形,$\angle OAP + \angle OPA = 0$,
同理,$\angle OBP + \angle OPB = 0$。于是
\begin{align}
    \angle AOB &= \angle AOQ + \angle QOB \notag \\ 
    &= \angle OAP + \angle APO + \angle PBO + \angle OPB \notag \\
    &= 2\angle APO + 2\angle OPB = 2\angle APB \notag
\end{align}
也就是说,圆心角是圆周角的两倍大小,圆周角是圆心角的一半大小。
由于每段圆弧只对应一个圆心角,无论$P$取圆上哪个点,圆周角$APB$都是圆心角的一半大小。

\begin{tm}\textbf{圆周角定理}\label{tm:0-1-0}
    给定圆$O$上的弧$\widearc{AB}$及圆上两点$P$、$Q$,
    $$\angle APB = \angle AQB = \frac{1}{2} \angle AOB.$$
\end{tm}
对径点和圆心形成平角,因此,根据圆周角定理,对径点对应的圆周角是直角。或者说,半圆对应的圆周角是直角。

同一个圆里,圆上的点$A$、$B$对应的圆心角$\angle AOB$和点$C$、$D$对应的圆心角$\angle COD$相等,那么
根据“边角边”,圆心$O$和它们构成的三角形满足:$\triangle AOB \simeq \triangle COD$。弦$AB$和$CD$也等长。
不仅如此,根据圆映射,圆弧$\widearc{AB}$和$\widearc{CD}$也等长。
事实上,$\widearc{CD}$就是$\widearc{AB}$关于某个角旋转的结果。
我们把这个结论称为“等角对等弦”、“等角对等弧”。

反之,如果两个圆弧$\widearc{AB}$和$\widearc{CD}$等长,那么它们对应的区间也一样长。
这说明它们对应的圆心角一样大。
圆心角既然相等,那么弦$AB$和$CD$也等长。
更进一步,设$P$是圆上的点,那么圆周角$\angle APB$和$\angle CPD$也一样大。我们把这个结论称为“等弧对等弦”、“等弧对等角”。

反过来,如果圆$O$上两条弦$AB$和$CD$等长,那么根据“边边边”,$\triangle AOB \simeq \triangle COD$。
于是圆心角相等,所以劣弧$\widearc{AB}$和$\widearc{CD}$等长。我们把这个结论称为“等弦对等角”、“等弦对等弧”。

总的来说,在同一个圆里,两点对应的弦长相等当且仅当对应的(劣弧)弧长相等,当且仅当对应的圆心角相等,
当且仅当对应的圆周角相等。弦、弧、圆心角、圆周角,都是用来描述圆的部分和整体关系的方法。

给定圆上两点$A$、$B$,它们对应的垂直平分线$l$平分$\angle AOB$,即把$\angle AOB$分成两个相同大小的圆心角。
因此,设$l$和圆交于$P$、$Q$,则它们也分别平分所在的圆弧(称为弧的中点)。
我们把这一系列结论总称为垂径定理:
\begin{tm}\textbf{垂径定理 }\label{tm:0-1-1}
    给定圆上两点,则恰有圆的一条直径垂直平分两点对应的弦,同时平分对应的圆心角和两个圆弧。
\end{tm}
垂径定理也可以说成:过圆$O$的弦$AB$中点的直径与弦$AB$垂直,同时平分$\angle AOB$和弧$\widearc{AB}$。

\begin{xt}\label{xt:0-1-0}
    给定圆$O$,弦$AB$中点记为$M$,$|MO|$称为弦$AB$的\textbf{弦心距}。\\
    \indent 1. 证明:圆心角相等,当且仅当对应的弦心距相等。\\
    \indent 2. 设直线$MO$与圆$O$交于$P$、$Q$两点,证明:$|MP| \cdot |MQ| = |MA|\cdot |MB|.$
\end{xt}

\section{圆内接多边形}
我们对圆上一点、两点引出的形状都有了初步了解,现在来看圆上多个点对应的形状。首先来看三个点的情形。

\section{弧长和面积}

% 由于角加上或减去若干个周角,仍然是同一个角,所以,数轴上的数和圆上的点之间也有映射。比如,从某个数:$783$出发,我们可以不断减去$360$,直到数值落在$0$和$360$之间。这时数值变成$63$,于是$783$对应到$63$度角对应的点。



\chapter{圆和三角形}
\section{圆幂}
\section{切线和割线}
\section{垂心和外接圆}
\section{内切圆和旁切圆}
\section{九点圆}

\chapter{三角函数}
\section{锐角的三角函数}
\section{三角函数的图像和性质}
\section{三角函数和三角形}


\chapter{从或许到确定}
\section{事件和试验}
\section{计数和概率}
\section{组合和排列}

\chapter{因式分解}
\section{因式和公因式}
\section{根和因式}

\chapter{三段论}
\section{大前提、小前提和结论}
\section{直言三段论}

\end{document}