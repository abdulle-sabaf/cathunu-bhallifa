\documentclass[12pt,UTF8]{ctexbook}
\usepackage{booktabs}
\usepackage{multirow}
\usepackage{subcaption}

% % 导入设定
% File settings - applied to all
% 导入第三方库
\usepackage{ctex}
\usepackage{array}
\usepackage{graphicx}
\usepackage{wrapfig}
\usepackage[table,dvipsnames]{xcolor}
\usepackage{tabularx}
\usepackage{longtable}
\usepackage{float}
\usepackage{amsmath}
\usepackage{amssymb}
\usepackage{mathtools}
\usepackage{polynom}
\usepackage{xfrac}
\usepackage{eucal}
\usepackage{titlesec}
\usepackage{amsthm}
\usepackage{mhchem}
\usepackage{tikz-cd}
\usepackage{enumitem}
\usepackage{verbatim}
\usepackage[makeroom]{cancel}
\usepackage[toc,page]{appendix}
\usepackage{fontspec,xunicode,xltxtra}
\usepackage{xeCJK} 
\usepackage{caption}
\usepackage[b]{esvect}
\usepackage{thmtools, thm-restate}
\usepackage{pifont}
\usepackage[perpage,symbol*]{footmisc}

% 修改脚注的编号为加圈样式,并且各页单独编号
\DefineFNsymbols{circled}{{\ding{192}}{\ding{193}}{\ding{194}}
{\ding{195}}{\ding{196}}{\ding{197}}{\ding{198}}{\ding{199}}{\ding{200}}{\ding{201}}}
\setfnsymbol{circled}

% 自定义颜色
\definecolor{gl}{RGB}{246, 252, 240}
\definecolor{gd}{RGB}{236, 244, 230}
\definecolor{bg}{RGB}{242, 244, 228}

% 定义字体
\setCJKmainfont[BoldFont=STZhongsong]{STSong}  % 普通字体、粗体
\setCJKmonofont{simkai.ttf} % \texttt
\setCJKsansfont{simfang.ttf} % \textsf

% 自制命令
\renewcommand{\thesection}{\arabic{chapter}.\arabic{section}}  % 章节使用阿拉伯数字
\renewcommand{\parallel}{\mathrel{/\mskip-4mu/}}  % 平行符号
\renewcommand{\proofname}{\indent\bf 证明}  % 自定义证明标题
\renewcommand{\qedsymbol}{\hfill$\square$}  % 自定义证毕符号
\newcommand{\e}{\mathrm{e}}  % 自然底数
\newcommand{\dash}{\,–\,}  % 短折号
\newcommand{\tong}[1]{\overset{#1}{\equiv\joinrel\equiv}}  % 同余等号
\newcommand{\di}[1]{\,\mathrm{d}#1}  % 微元d
\newcommand{\qu}[2]{\displaystyle\left(#1;#2\right)}  % 开区间
% 局部展开 developpements limites
\newcommand{\oveq}[1]{\overset{#1}{=}}   % equal over
\newcommand{\olim}[1]{\mathit{o}\left(#1\right)}  % petit o
\newcommand{\Olim}[1]{\mathcal{O}\left(#1\right)}  % grand O
\newcommand{\Tlim}[1]{\mathcal{\Theta}\left(#1\right)}  % grand theta
\newcommand{\eqlim}[1]{\overset{#1}{\sim}}  % equivalence
\newcommand{\vect}[1]{\left\langle #1 \right\rangle}  % 生成空间 generated space

\newcommand{\arccot}{\operatorname{arccot}}  % 反余弦函数
\newcommand{\dlim}[1]{^{\color{gray}\prime}#1}  % 数字分隔符
\newcommand{\lian}[1]{  % 极限符号
    \underset{#1}{\operatorname{lian}\,}
}
\newcommand{\nji}[2]{\displaystyle\left( #1 \,|\, #2 \right)}  % 内积
\newcommand{\dangle}{  % 角符号
    \mathord{
        \text{  %
            \tikz[baseline] \draw (0.8em,0ex) -- (0.3em, 0ex) -- (.6em, 1.5ex) -- (.8em, 1.5ex) -- (.5em, 0ex) -- cycle;
        }
    }
}
\newcommand{\xangle}{  % 角符号
    \mathord{
        \text{%
        \tikz[baseline] \draw (0.8em,1.5ex) -- (0.3em, 0ex) -- (.64em, 0ex) -- (.8em, .36ex) -- (.42em, .36ex) -- cycle;
        }
    }
}
\newcommand{\bu}{  % 补集符号
    \mathbin{
        \text{
            \tikz[baseline=-0.6ex]{
                \node[draw, fill=black, minimum size=0.8ex, inner sep=0pt, rectangle] (bu) {};
                \node[draw=none, fill=white, minimum size=0.6ex, inner sep=0pt, circle] at (bu.center) {};
            }
        }
    }
}
\newcommand{\rectbx}{  % 长方形符号
    \mathord{
        \text{%
            \tikz[baseline] \draw (0,.1ex) -- (.4em,.1ex) -- (.4em,1.5ex) -- (0em,1.5ex) -- cycle;
        }
    }
}
\newcommand{\tr}{  % 矩阵转置符号 A^{\tr} 
    \mathord{
        \begin{tikzpicture}[baseline=-0.2em, line width=0.3pt]
        \draw (-0.15em, 0.15em) -- (0.06em, -0.06em);
        \draw (45:0.15em) arc[start angle=45, end angle=225, radius=0.15em];
    \end{tikzpicture}
    }
}
\newcommand{\arcangle}{\mathord{\mathpalette\doarcangle\relax}}  % 带弧的角度符号 - 交角
\newcommand{\doarcangle}[2]{  % 
    \hbox{%
        \sbox0{$#1B$}%
        \sbox2{$#1<$}%
        \raisebox{\dimexpr\dp0+(\ht0-\ht2)/2}{%
            $#1<\mspace{-9mu}\mathrel{)}\mspace{2mu}$%
        }%
    }%
}
\newcommand{\parasbx}{  % 平行四边形符号
    \mathord{
        \text{%
            \tikz[baseline] \draw (0,.1ex) -- (.8em,.1ex) -- (1em,1.6ex) -- (.2em,1.6ex) -- cycle;
        }
    }
}
\usetikzlibrary{calc,topaths}
\newcommand{\widearc}[1]{  % 可伸缩圆弧符号
    \tikz[baseline=(wideArcAnchor.base)]{
        \node[inner sep=0] (wideArcAnchor) {$#1$}; 
        \coordinate (wideArcAnchorA) at ($(wideArcAnchor.north west) + (0.15em,0.1em)$);
        \coordinate (wideArcAnchorB) at ($(wideArcAnchor.north east) + (0.0em,0.1em)$);
        \draw[line width=0.1ex,line cap=round,out=45,in=135] (wideArcAnchorA) to (wideArcAnchorB);
    }
}

% 定义、定理、证明等块环境
\theoremstyle{definition}
\newtheorem{df}{定义}[section] 
\newtheorem*{po}{公理}
\newtheorem{pp}{命题}[section]
\newtheorem{tm}{定理}[section]
\newtheorem{cor}{推论}[pp]
\newtheorem{ex}{例子}[section]
\newtheorem{et}{例题}[section]
\newtheorem*{ex*}{例子}
\newtheorem*{so}{解答}
\theoremstyle{plain}
\newtheorem{sk}{思考}[section]
\newtheorem{xt}{习题}[section]
\renewenvironment{proof}{\paragraph{\textbf{证明:}}}{\hfill$\square$}
% \declaretheorem[name=定义, numberwithin=section, shaded={rulecolor={rgb}{0.1,0.7,0.4},
% rulewidth=2pt, bgcolor={rgb}{0.96,1,0.99}}]{df}
% \declaretheorem[name=定理, numberwithin=section, shaded={rulecolor={rgb}{0.1,0.4,0.7},
% rulewidth=2pt, bgcolor={rgb}{0.96,0.99,1}}]{tm}
% \declaretheorem[name=思考, numberwithin=section, shaded={rulecolor={rgb}{0,0.7,0.7},
% rulewidth=2pt, bgcolor={rgb}{0.98,1,1}}]{sk}
% \declaretheorem[name=习题, numberwithin=section, shaded={rulecolor={rgb}{0.91,0.84,0.42},
% rulewidth=2pt, bgcolor={rgb}{1,0.98,0.93}}]{xt}

\setlength{\intextsep}{2pt}%
\setlength{\columnsep}{2pt}%
% 列举环境
\setlist{label=\textbullet}
% 列举环境行间距
\setenumerate[1]{itemsep=0pt,partopsep=0pt,parsep=0pt,topsep=0pt}
\setitemize[1]{itemsep=0pt,partopsep=0pt,parsep=0pt,topsep=0pt}
\setdescription{itemsep=0pt,partopsep=0pt,parsep=0pt,topsep=0pt}
% 章节间距
\setlength\parskip{8pt}
% 文本框间距
\setlength{\fboxsep}{12pt}
% 章节字体大小
\titleformat{\section}{\zihao{-2}\bfseries}{ \thesection }{16pt}{}

% % 封面
% \title{\zihao{0} \bfseries 测试页}
% \author{\zihao{2} \texttt{大青花鱼}}
% % \date{\bfseries\today}
% \date{}
% % 正文
% \begin{document}
% \maketitle
% \tableofcontents
% \newpage
\usepackage{tikz}
\usepackage{pgfplots}
\pgfplotsset{compat=1.18}
\usetikzlibrary{calc,arrows.meta}

\begin{document}

% \section{问题背景}
% 研究某城市25-35岁居民的教育年限($Y$,单位:年)与年收入($X$,单位:万元)的关系。随机抽样1000名居民,得到联合分布数据如下:

% \section{数据分布}
% \begin{table}[h!]
% \centering
% \caption{教育年限与收入的联合频数分布(样本量$n=1000$)}
% \label{tab:joint-dist}
% \begin{tabular}{c|ccccc|c}
% \toprule
% \multirow{2}{*}{$X$\textbackslash $Y$} & \multicolumn{4}{c}{教育年限 $Y$ (年)} & \multirow{2}{*}{合计} \\
% & 12 & 16 & 18 & 20 & \\
% \midrule
% 10 & 140 & 40 & 10 & 0 & 190 \\
% 20 & 90 & 140 & 40 & 10 & 280 \\
% 30 & 40 & 90 & 100 & 40 & 270 \\
% 40 & 10 & 40 & 80 & 60 & 190 \\
% 50 & 0 & 10 & 20 & 40 & 70 \\
% \hline
% 合计 & 280 & 320 & 250 & 150 & 1000 \\
% \bottomrule
% \end{tabular}
% \end{table}

% \section{基于事件的条件分布}
% 定义随机事件:
% \[
% A_1 = \{Y \leq 16\} \quad (\text{高等教育以下}), \quad 
% A_2 = \{Y > 16\} \quad (\text{高等教育以上})
% \]

% \subsection{条件概率计算}
% \begin{align*}
% P(A_1) &= \frac{280 + 320}{1000} = 0.6, \quad 
% P(A_2) = \frac{250 + 150}{1000} = 0.4 \\
% P(X=x|A_1) &= \frac{P(X=x, Y\leq16)}{P(A_1)}, \quad 
% P(X=x|A_2) = \frac{P(X=x, Y>16)}{P(A_2)}
% \end{align*}

% \begin{table}[h!]
% \centering
% \caption{条件分布 $P(X|A_1)$ 和 $P(X|A_2)$}
% \label{tab:event-cond}
% \begin{tabular}{c|cc|cc}
% \toprule
% \multirow{2}{*}{收入 $X$} & \multicolumn{2}{c|}{频数} & \multicolumn{2}{c}{条件概率} \\
% & $A_1$ & $A_2$ & $P(X|A_1)$ & $P(X|A_2)$ \\
% \midrule
% 10 & 180 & 10 & 0.300 & 0.025 \\
% 20 & 230 & 50 & 0.383 & 0.125 \\
% 30 & 130 & 140 & 0.217 & 0.350 \\
% 40 & 50 & 140 & 0.083 & 0.350 \\
% 50 & 10 & 60 & 0.017 & 0.150 \\
% \hline
% 合计 & 600 & 400 & 1.000 & 1.000 \\
% \bottomrule
% \end{tabular}
% \end{table}

% \subsection{分布可视化}
% \begin{figure}[h!]
% \centering
% \begin{tikzpicture}
% \begin{axis}[
%     ybar, 
%     bar width=6mm,
%     title={事件条件对收入分布的影响},
%     xlabel={收入 \(X\) (万元)},
%     ylabel={概率},
%     symbolic x coords={10,20,30,40,50},
%     xtick=data,
%     legend pos=north west,
%     ymajorgrids=true,
%     width=0.9\textwidth,
%     height=0.6\textwidth
% ]
% \addplot coordinates {(10,0.300) (20,0.383) (30,0.217) (40,0.083) (50,0.017)};
% \addplot coordinates {(10,0.025) (20,0.125) (30,0.350) (40,0.350) (50,0.150)};
% \legend{\(X|A_1\) (教育≤16年), \(X|A_2\) (教育>16年)}
% \end{axis}
% \end{tikzpicture}
% \caption{事件条件对收入分布的影响}
% \label{fig:event-cond}
% \end{figure}

% \section{基于随机变量的条件分布}
% 定义条件随机变量 $X|Y=y$ 的概率质量函数:
% \[
% P(X=x|Y=y) = \frac{P(X=x,Y=y)}{P(Y=y)}
% \]

% \begin{table}[h!]
% \centering
% \caption{条件分布 $P(X|Y=y)$ 随教育年限 $y$ 的变化}
% \label{tab:rv-cond}
% \begin{tabular}{c|cccc}
% \toprule
% \multirow{2}{*}{收入 $X$} & \multicolumn{4}{c}{教育年限 $Y=y$ (年)} \\
% & 12 & 16 & 18 & 20 \\
% \midrule
% 10 & 0.500 & 0.125 & 0.040 & 0.000 \\
% 20 & 0.321 & 0.438 & 0.160 & 0.067 \\
% 30 & 0.143 & 0.281 & 0.400 & 0.267 \\
% 40 & 0.036 & 0.125 & 0.320 & 0.400 \\
% 50 & 0.000 & 0.031 & 0.080 & 0.267 \\
% \hline
% 合计 & 1.000 & 1.000 & 1.000 & 1.000 \\
% \bottomrule
% \end{tabular}
% \end{table}

% \subsection{分布可视化}
% \begin{figure}[h!]
% \centering
% \begin{tikzpicture}
% \begin{axis}[
%     ybar, 
%     bar width=4mm,
%     title={不同教育年限下的收入分布},
%     xlabel={收入 \(X\) (万元)},
%     ylabel={概率},
%     symbolic x coords={10,20,30,40,50},
%     xtick=data,
%     legend style={at={(0.5,-0.2)}, anchor=north, legend columns=4},
%     ymajorgrids=true,
%     width=1.0\textwidth,
%     height=0.6\textwidth
% ]
% \addplot coordinates {(10,0.500) (20,0.321) (30,0.143) (40,0.036) (50,0.000)};
% \addplot coordinates {(10,0.125) (20,0.438) (30,0.281) (40,0.125) (50,0.031)};
% \addplot coordinates {(10,0.040) (20,0.160) (30,0.400) (40,0.320) (50,0.080)};
% \addplot coordinates {(10,0.000) (20,0.067) (30,0.267) (40,0.400) (50,0.267)};
% \legend{\(Y=12\), \(Y=16\), \(Y=18\), \(Y=20\)}
% \end{axis}
% \end{tikzpicture}
% \caption{不同教育水平下的条件收入分布}
% \label{fig:rv-cond}
% \end{figure}

% \section{关键结论}
% \begin{enumerate}
% \item \textbf{事件条件改变分布形状}:从图\ref{fig:event-cond}可见:
%   \begin{itemize}
%   \item $X|A_1$ 呈右偏分布(众数20万)
%   \item $X|A_2$ 呈钟形分布(众数30-40万)
%   \end{itemize}
  
% \item \textbf{随机变量条件精细调节分布}:从图\ref{fig:rv-cond}可见:
%   \begin{align*}
%   \text{教育水平 } \uparrow \quad \Rightarrow \quad & \text{低收入概率} \downarrow \\
%   & \text{高收入概率} \uparrow \\
%   & \text{分布中心} \rightarrow \text{右移}
%   \end{align*}
  
% \item \textbf{期望收入对比}:
%   \begin{align*}
%   E[X|A_1] &= 21.8 \text{ 万元}, \quad 
%   E[X|A_2] = 33.5 \text{ 万元} \\
%   E[X|Y=12] &= 18.2 \text{ 万元}, \quad 
%   E[X|Y=16] = 24.8 \text{ 万元} \\
%   E[X|Y=18] &= 32.4 \text{ 万元}, \quad 
%   E[X|Y=20] = 37.3 \text{ 万元}
%   \end{align*}
% \end{enumerate}

% \section{理论启示}
% \begin{itemize}
% \item \textbf{事件条件} ($X|A$):生成\textbf{单一}条件分布,通过事件$A$改变样本空间
% \item \textbf{随机变量条件} ($X|Y=y$):生成\textbf{分布族},展示$X$如何随$Y$取值变化
% \item \textbf{分布演化}:当$Y$连续时,$X|Y=y$可形成条件密度曲面
% \item \textbf{应用意义}:条件期望$E[X|Y]$是最小均方预测器
% \end{itemize}


% \begin{tikzpicture}[scale=0.75]
%     % 方城参数
%     \pgfmathsetmacro{\citySize}{4}    % 方城边长
%     \pgfmathsetmacro{\treeNorth}{1}   % 北门出城100步
%     \pgfmathsetmacro{\walkSouth}{1}   % 南门出城100步
%     \pgfmathsetmacro{\walkWest}{12}    % 西行1200步
%     \pgfmathsetmacro{\door}{0.24}    % 城门用

%     % 绘制坐标轴
%     \draw[->] (-12.8,0) -- (3,0) node[below right] {$x(100\mbox{\texttt{步}})$};
%     \draw[->] (0,-3.5) -- (0,4.8) node[below right] {$y(100\mbox{\texttt{步}})$};

%     \foreach \x in {-12,...,-1} {
%         \draw (\x,0) -- (\x,-0.1);
%         \node [below] at (\x,0) {\x};
%     }
    
%     \foreach \x in {1,2} {
%         \draw (\x,0) -- (\x,-0.1);
%         \node [below] at (\x,0) {\x};
%     }

%     \foreach \y in {-3,...,-1} {
%         \draw (0,\y) -- (-0.1,\y);
%         \node [left] at (0,\y) {\y};
%     }

%     \foreach \y in {1,...,4} {
%         \draw (0,\y) -- (-0.1,\y);
%         \node [left] at (0,\y) {\y};
%     }

%     % 绘制方城(以原点为中心)
%     \draw[thick, draw=Sepia] (-\citySize/2, -\citySize/2) rectangle (\citySize/2, \citySize/2);
%     \draw[thick, draw=Sepia] (-\door/2, \citySize/2) -- (-\door/2, \citySize/2 + \door/2);
%     \draw[thick, draw=Sepia] (\door/2, \citySize/2) -- (\door/2, \citySize/2 + \door/2);
%     \draw[thick, draw=Sepia] (-\door/2, -\citySize/2) -- (-\door/2, -\citySize/2 - \door/2);
%     \draw[thick, draw=Sepia] (\door/2, -\citySize/2) -- (\door/2, -\citySize/2 - \door/2);
%     \draw[thick, draw=Sepia] (\citySize/2, -\door/2) -- (\citySize/2 + \door/2, -\door/2);
%     \draw[thick, draw=Sepia] (\citySize/2, \door/2) -- (\citySize/2 + \door/2, \door/2);
%     \draw[thick, draw=Sepia] (-\citySize/2, -\door/2) -- (-\citySize/2 - \door/2, -\door/2);
%     \draw[thick, draw=Sepia] (-\citySize/2, \door/2) -- (-\citySize/2 - \door/2, \door/2);
    
%     % 绘制树
%     \node[draw, circle, fill=green!50!black, inner sep=2pt] at (0, \citySize/2 + \treeNorth) {};
%     \node[above] at (0, \citySize/2 + \treeNorth) {};
    
%     % 绘制南门出城路径
%     \draw[-{Latex[length=2.5mm,width=2mm]}, thick, draw=red] (0, \citySize/2) -- node[right] {\texttt{北门出城直走}$100$\texttt{步}}  (0, \citySize/2 + \walkSouth);
%     \draw[thick, draw=red] (0, -\citySize/2) -- (0, -\citySize/2 - \walkSouth);
%     \draw[-{Latex[length=2.5mm,width=2mm]}, thick, draw=red] (0, -\citySize/2 - \walkSouth) -- node[above] {\texttt{南门出城直走}$100$\texttt{步转西走}$1200$\texttt{步}} (-\walkWest, -\citySize/2 - \walkSouth);
    
%     % 绘制视角线
%     \draw[dashed] (0, \citySize/2 + \treeNorth) -- (-\walkWest, -\citySize/2 - \walkSouth);

%     \draw[-{Stealth[length=6mm,width=3mm]}, thick] (-10,3) -- (-10, 3.5);
%     \node[above] at (-10, 3.5) {\texttt{北}};
    
% \end{tikzpicture}

% \newpage

% \begin{align*}
%     22 \times 3000 \times b + 50000 \times (m - 5) &= 66000\cdot \left(-m - 1 - \sum_{i=0}^{5+m} \frac{i\e^{-4}4^i}{i!} + (5+m) \sum_{i=0}^{5} \frac{\e^{-4}4^i}{i!} \right) + 50000 \cdot (m - 5) \\
%     &= 6\dlim{6000}\cdot \left((5+m) \sum_{i=0}^{5} \frac{\e^{-4}4^i}{i!} - \sum_{i=0}^{5+m} \frac{i\e^{-4}4^i}{i!}\right) -1\dlim{6000} m - 31\dlim{6000}  
% \end{align*}

% \begin{tikzpicture}
%     \begin{axis}[
%         width=12cm, height=6cm,
%         xlabel={},
%         ylabel={\texttt{概率}},
%         xtick={1,2,...,12},
%         ytick={0,0.05,...,0.4},
%         ymin=0, ymax=0.45,
%         xmin=0, xmax=12,
%         grid=both,
%         grid style={line width=.1pt, draw=gray!30, dashed},
%         axis lines=middle,
%         enlargelimits={upper=0.1},
%         yticklabel style={
%             /pgf/number format/fixed,
%             /pgf/number format/precision=2
%         }
%     ]
    
%     % 定义二项分布的概率质量函数
%     \pgfmathdeclarefunction{binomial}{1}{%
%       \pgfmathparse{(0.4) * ((1 - 0.4)^(\x - 1))}%
%     }
    
%     \addplot+[ybar, samples at={1,2,...,12}, fill=Sepia!30] 
%         (\x, {binomial(\x)});
        
%     \end{axis}
    
%     \node [below] at (10.2cm,-0.18cm) {$\ldots$};

%     \node [above] at (9.6cm,0.14cm) {\texttt{随机变量值}};

%     \node [below, align=center] at (6cm, -1cm) {\texttt{系数}$p=0.4$\texttt{的等比分布}};

% \end{tikzpicture}
    

% \newpage

% \begin{tikzpicture}
%     \begin{axis}[
%         width=12cm, height=6cm,
%         xlabel={},
%         ylabel={\texttt{概率}},
%         xtick={0,2,...,20},
%         ytick={0,0.02,...,0.12},
%         ymin=0, ymax=0.12,
%         xmin=0, xmax=20,
%         yticklabel style={
%             /pgf/number format/fixed,
%             /pgf/number format/precision=2
%         },
%         grid=both,
%         grid style={line width=.1pt, draw=gray!30, dashed},
%         axis lines=middle,
%         enlargelimits={upper=0.1}
%     ]
    
%     % 定义泊松分布的概率质量函数
%     \pgfmathdeclarefunction{poisson}{1}{%
%       \pgfmathparse{exp(-#1) * (#1^\x) / \x!}%
%     }
    
%     \addplot+[ybar, samples at={0,...,20}, fill=Sepia!30] 
%         (\x, {poisson(10)});
    
%     \end{axis}
    
%     \node [below] at (10.2cm,-0.18cm) {$\ldots$};

%     \node [above] at (9.6cm,0.14cm) {\texttt{随机变量值}};

%     \node [below, align=center] at (6cm, -1cm) {\texttt{系数}$a=10$\texttt{的指数分布}};
    
% \end{tikzpicture}

% \begin{tikzpicture}

% \def\maxDots{12}
% \def\maxRollValue{6}

% \def\xunit{0.9cm}
% \def\yunit{0.5cm}
% \def\xdif{0.9}
% \def\ydif{0.5}

% \draw [->,line width=1pt,>={Stealth[]}] (0,0) -- (\maxDots*\xunit,0) node[above left] {总点数};
% \draw [->,line width=1pt,>={Stealth[]}] (0,0) -- (0,\maxRollValue*\yunit+1.5*\yunit);

% \foreach \x in {2,...,\maxDots} {
%     \draw (\x*\xunit-\xunit,0) -- (\x*\xunit-\xunit,-0.1*\yunit);
%     \node [below] at (\x*\xunit-\xunit,0) {\x};
% }

% \foreach \y in {1,...,\maxRollValue} {
%     \draw (0,\y*\yunit) -- (-0.1*\xunit,\y*\yunit);
%     \node [left] at (0,\y*\yunit) {\y};
% }

% \node [below left] at (0,0) {O};

% \foreach \i in {1,...,\maxRollValue} {
%     \foreach \j in {\i,...,\maxRollValue} {
%         \pgfmathsetmacro{\maxVal}{max(\i,\j)}
%         \pgfmathsetmacro{\minVal}{min(\i,\j)}
%         \pgfmathsetmacro{\sumVal}{\i+\j}
        
%         \draw [red!80!black, fill=red!10] ($(\sumVal*\xunit-\xunit,\maxVal*\yunit) + (-0.4*\xunit, -0.4*\yunit)$) rectangle ($(\sumVal*\xunit-\xunit,\maxVal*\yunit) + (0.4*\xunit, 0.4*\yunit)$);
%         \draw [red!80!black, fill=red!10] ($(\sumVal*\xunit-\xunit,\minVal*\yunit) + (-0.4*\xunit, -0.4*\yunit)$) rectangle ($(\sumVal*\xunit-\xunit,\minVal*\yunit) + (0.4*\xunit, 0.4*\yunit)$);
%         \node [font=\scriptsize] at (\sumVal*\xunit-\xunit,\maxVal*\yunit) {(\i,\j)};
%         \node [font=\scriptsize] at (\sumVal*\xunit-\xunit,\minVal*\yunit) {(\j,\i)};
%     }
% }

% % 新增直方图部分
% \begin{scope}[yshift=-4cm]
%     \foreach \sumVal/\probability in {2/1,3/2,4/3,5/4,6/5,7/6,8/5,9/4,10/3,11/2,12/1} {
%         \pgfmathsetmacro{\barHeight}{\probability*\ydif}
%         \draw [red!80!black, fill=red!20, fill opacity=0.5] (\sumVal*\xdif-\xdif-0.4*\xdif,0) rectangle (\sumVal*\xdif+0.4*\xdif-\xdif,\barHeight);
%         % \node [below] at (\sumVal*\xdif-\xdif, -0.2*\ydif) {\sumVal};
%         % \node [left] at (\sumVal*\xdif-0.6*\xdif, \barHeight/2) {\probability};
%     }
%     \draw [->, line width=1pt, >={Stealth[]}, line cap=round] (0,0) -- (\maxDots*\xunit, 0);
%     \draw [line width=1pt, >={Stealth[]}, line cap=round] (0,0) -- (0, \maxRollValue*\yunit + 3*\yunit);
%     \node [below left] at (0,0) {O};
%     \foreach \x in {2,...,\maxDots} {
%         \draw (\x*\xunit-\xunit,0) -- (\x*\xunit-\xunit,-0.1*\yunit);
%         \node [below] at (\x*\xunit-\xunit,0) {\x};
%     }
    
%     \foreach \y in {1,...,\maxRollValue} {
%         \draw (0,\y*\yunit) -- (-0.1*\xunit,\y*\yunit);
%         \node [left] at (0,\y*\yunit) {\y/36};
%     }
% \end{scope}

% \end{tikzpicture}

% \(\bmat{f}[B] \touying{(\mathbf{a}, \mathbf{b})}, \quad \fanshe{(\mathbf{a}, \mathbf{b})}, \quad \xuan{(\mathbf{a}, \mathbf{b})}\)

% \begin{tikzpicture}[scale=1, tdplot_main_coords]
%     \def\n{5} % 设置 n=3

%     % 绘制网格(可选)
%     \foreach \x in {0,1,...,\n}
%         \draw[grid3, thin] (\x,0,0) -- (\x,\n,0);
%     \foreach \x in {0,1,...,\n}
%         \draw[grid3, thin] (\x,0,0) -- (\x,0,\n);
%     \foreach \y in {0,1,...,\n}
%         \draw[grid3, thin] (0,\y,0) -- (\n,\y,0);
%     \foreach \y in {0,1,...,\n}
%         \draw[grid3, thin] (0,\y,0) -- (0,\y,\n);
%     \foreach \z in {0,1,...,\n}
%         \draw[grid3, thin] (0,0,\z) -- (0,\n,\z);
%     \foreach \z in {0,1,...,\n}
%         \draw[grid3, thin] (0,0,\z) -- (\n,0,\z);
%     % 绘制坐标轴
%     \draw[thick,->] (0,0,0) -- (\n+1,0,0) node[anchor=north east]{$x$};
%     \draw[thick,->] (0,0,0) -- (0,\n+1,0) node[anchor=north west]{$y$};
%     \draw[thick,->] (0,0,0) -- (0,0,\n+1) node[anchor=south]{$z$};
% \end{tikzpicture}

\end{document}