% File settings - applied to all
% 导入第三方库
\usepackage{ctex}
\usepackage{array}
\usepackage{graphicx}
\usepackage{wrapfig}
\usepackage[table,dvipsnames]{xcolor}
\usepackage{tabularx}
\usepackage{longtable}
\usepackage{float}
\usepackage{amsmath}
\usepackage{amssymb}
\usepackage{mathtools}
\usepackage{polynom}
\usepackage{xfrac}
\usepackage{eucal}
\usepackage{titlesec}
\usepackage{amsthm}
\usepackage{mhchem}
\usepackage{tikz-cd}
\usepackage{enumitem}
\usepackage{verbatim}
\usepackage[makeroom]{cancel}
\usepackage[toc,page]{appendix}
\usepackage{fontspec,xunicode,xltxtra}
\usepackage{xeCJK} 
\usepackage{caption}
\usepackage[b]{esvect}
\usepackage{thmtools, thm-restate}
\usepackage{pifont}
\usepackage[perpage,symbol*]{footmisc}

% 修改脚注的编号为加圈样式,并且各页单独编号
\DefineFNsymbols{circled}{{\ding{192}}{\ding{193}}{\ding{194}}
{\ding{195}}{\ding{196}}{\ding{197}}{\ding{198}}{\ding{199}}{\ding{200}}{\ding{201}}}
\setfnsymbol{circled}

% 自定义颜色
\definecolor{gl}{RGB}{246, 252, 240}
\definecolor{gd}{RGB}{236, 244, 230}
\definecolor{bg}{RGB}{242, 244, 228}

% 定义字体
\setCJKmainfont[BoldFont=STZhongsong]{STSong}  % 普通字体、粗体
\setCJKmonofont{simkai.ttf} % \texttt
\setCJKsansfont{simfang.ttf} % \textsf

% 自制命令
\renewcommand{\thesection}{\arabic{chapter}.\arabic{section}}  % 章节使用阿拉伯数字
\renewcommand{\parallel}{\mathrel{/\mskip-4mu/}}  % 平行符号
\renewcommand{\proofname}{\indent\bf 证明}  % 自定义证明标题
\renewcommand{\qedsymbol}{\hfill$\square$}  % 自定义证毕符号
\newcommand{\e}{\mathrm{e}}  % 自然底数
\newcommand{\dash}{\,–\,}  % 短折号
\newcommand{\tong}[1]{\overset{#1}{\equiv\joinrel\equiv}}  % 同余等号
\newcommand{\di}[1]{\,\mathrm{d}#1}  % 微元d
\newcommand{\qu}[2]{\displaystyle\left(#1;#2\right)}  % 开区间
% 局部展开 developpements limites
\newcommand{\oveq}[1]{\overset{#1}{=}}   % equal over
\newcommand{\olim}[1]{\mathit{o}\left(#1\right)}  % petit o
\newcommand{\Olim}[1]{\mathcal{O}\left(#1\right)}  % grand O
\newcommand{\Tlim}[1]{\mathcal{\Theta}\left(#1\right)}  % grand theta
\newcommand{\eqlim}[1]{\overset{#1}{\sim}}  % equivalence
\newcommand{\vect}[1]{\left\langle #1 \right\rangle}  % 生成空间 generated space

\newcommand{\arccot}{\operatorname{arccot}}  % 反余弦函数
\newcommand{\dlim}[1]{^{\color{gray}\prime}#1}  % 数字分隔符
\newcommand{\lian}[1]{  % 极限符号
    \underset{#1}{\operatorname{lian}\,}
}
\newcommand{\nji}[2]{\displaystyle\left( #1 \,|\, #2 \right)}  % 内积
\newcommand{\dangle}{  % 角符号
    \mathord{
        \text{  %
            \tikz[baseline] \draw (0.8em,0ex) -- (0.3em, 0ex) -- (.6em, 1.5ex) -- (.8em, 1.5ex) -- (.5em, 0ex) -- cycle;
        }
    }
}
\newcommand{\xangle}{  % 角符号
    \mathord{
        \text{%
        \tikz[baseline] \draw (0.8em,1.5ex) -- (0.3em, 0ex) -- (.64em, 0ex) -- (.8em, .36ex) -- (.42em, .36ex) -- cycle;
        }
    }
}
\newcommand{\bu}{  % 补集符号
    \mathbin{
        \text{
            \tikz[baseline=-0.6ex]{
                \node[draw, fill=black, minimum size=0.8ex, inner sep=0pt, rectangle] (bu) {};
                \node[draw=none, fill=white, minimum size=0.6ex, inner sep=0pt, circle] at (bu.center) {};
            }
        }
    }
}
\newcommand{\rectbx}{  % 长方形符号
    \mathord{
        \text{%
            \tikz[baseline] \draw (0,.1ex) -- (.4em,.1ex) -- (.4em,1.5ex) -- (0em,1.5ex) -- cycle;
        }
    }
}
\newcommand{\tr}{  % 矩阵转置符号 A^{\tr} 
    \mathord{
        \begin{tikzpicture}[baseline=-0.2em, line width=0.3pt]
        \draw (-0.15em, 0.15em) -- (0.06em, -0.06em);
        \draw (45:0.15em) arc[start angle=45, end angle=225, radius=0.15em];
    \end{tikzpicture}
    }
}
\newcommand{\arcangle}{\mathord{\mathpalette\doarcangle\relax}}  % 带弧的角度符号 - 交角
\newcommand{\doarcangle}[2]{  % 
    \hbox{%
        \sbox0{$#1B$}%
        \sbox2{$#1<$}%
        \raisebox{\dimexpr\dp0+(\ht0-\ht2)/2}{%
            $#1<\mspace{-9mu}\mathrel{)}\mspace{2mu}$%
        }%
    }%
}
\newcommand{\parasbx}{  % 平行四边形符号
    \mathord{
        \text{%
            \tikz[baseline] \draw (0,.1ex) -- (.8em,.1ex) -- (1em,1.6ex) -- (.2em,1.6ex) -- cycle;
        }
    }
}
\usetikzlibrary{calc,topaths}
\newcommand{\widearc}[1]{  % 可伸缩圆弧符号
    \tikz[baseline=(wideArcAnchor.base)]{
        \node[inner sep=0] (wideArcAnchor) {$#1$}; 
        \coordinate (wideArcAnchorA) at ($(wideArcAnchor.north west) + (0.15em,0.1em)$);
        \coordinate (wideArcAnchorB) at ($(wideArcAnchor.north east) + (0.0em,0.1em)$);
        \draw[line width=0.1ex,line cap=round,out=45,in=135] (wideArcAnchorA) to (wideArcAnchorB);
    }
}

% 定义、定理、证明等块环境
\theoremstyle{definition}
\newtheorem{df}{定义}[section] 
\newtheorem*{po}{公理}
\newtheorem{pp}{命题}[section]
\newtheorem{tm}{定理}[section]
\newtheorem{cor}{推论}[pp]
\newtheorem{ex}{例子}[section]
\newtheorem{et}{例题}[section]
\newtheorem*{ex*}{例子}
\newtheorem*{so}{解答}
\theoremstyle{plain}
\newtheorem{sk}{思考}[section]
\newtheorem{xt}{习题}[section]
\renewenvironment{proof}{\paragraph{\textbf{证明:}}}{\hfill$\square$}
% \declaretheorem[name=定义, numberwithin=section, shaded={rulecolor={rgb}{0.1,0.7,0.4},
% rulewidth=2pt, bgcolor={rgb}{0.96,1,0.99}}]{df}
% \declaretheorem[name=定理, numberwithin=section, shaded={rulecolor={rgb}{0.1,0.4,0.7},
% rulewidth=2pt, bgcolor={rgb}{0.96,0.99,1}}]{tm}
% \declaretheorem[name=思考, numberwithin=section, shaded={rulecolor={rgb}{0,0.7,0.7},
% rulewidth=2pt, bgcolor={rgb}{0.98,1,1}}]{sk}
% \declaretheorem[name=习题, numberwithin=section, shaded={rulecolor={rgb}{0.91,0.84,0.42},
% rulewidth=2pt, bgcolor={rgb}{1,0.98,0.93}}]{xt}

\setlength{\intextsep}{2pt}%
\setlength{\columnsep}{2pt}%
% 列举环境
\setlist{label=\textbullet}
% 列举环境行间距
\setenumerate[1]{itemsep=0pt,partopsep=0pt,parsep=0pt,topsep=0pt}
\setitemize[1]{itemsep=0pt,partopsep=0pt,parsep=0pt,topsep=0pt}
\setdescription{itemsep=0pt,partopsep=0pt,parsep=0pt,topsep=0pt}
% 章节间距
\setlength\parskip{8pt}
% 文本框间距
\setlength{\fboxsep}{12pt}
% 章节字体大小
\titleformat{\section}{\zihao{-2}\bfseries}{ \thesection }{16pt}{}