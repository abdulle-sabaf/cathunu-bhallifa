\documentclass[12pt,UTF8]{ctexbook}

% 导入设定
% File settings - applied to all
% 导入第三方库
\usepackage{ctex}
\usepackage{array}
\usepackage{graphicx}
\usepackage{wrapfig}
\usepackage[table,dvipsnames]{xcolor}
\usepackage{tabularx}
\usepackage{longtable}
\usepackage{float}
\usepackage{amsmath}
\usepackage{amssymb}
\usepackage{mathtools}
\usepackage{polynom}
\usepackage{xfrac}
\usepackage{eucal}
\usepackage{titlesec}
\usepackage{amsthm}
\usepackage{mhchem}
\usepackage{tikz-cd}
\usepackage{enumitem}
\usepackage{verbatim}
\usepackage[makeroom]{cancel}
\usepackage[toc,page]{appendix}
\usepackage{fontspec,xunicode,xltxtra}
\usepackage{xeCJK} 
\usepackage{caption}
\usepackage[b]{esvect}
\usepackage{thmtools, thm-restate}
\usepackage{pifont}
\usepackage[perpage,symbol*]{footmisc}

% 修改脚注的编号为加圈样式,并且各页单独编号
\DefineFNsymbols{circled}{{\ding{192}}{\ding{193}}{\ding{194}}
{\ding{195}}{\ding{196}}{\ding{197}}{\ding{198}}{\ding{199}}{\ding{200}}{\ding{201}}}
\setfnsymbol{circled}

% 自定义颜色
\definecolor{gl}{RGB}{246, 252, 240}
\definecolor{gd}{RGB}{236, 244, 230}
\definecolor{bg}{RGB}{242, 244, 228}

% 定义字体
\setCJKmainfont[BoldFont=STZhongsong]{STSong}  % 普通字体、粗体
\setCJKmonofont{simkai.ttf} % \texttt
\setCJKsansfont{simfang.ttf} % \textsf

% 自制命令
\renewcommand{\thesection}{\arabic{chapter}.\arabic{section}}  % 章节使用阿拉伯数字
\renewcommand{\parallel}{\mathrel{/\mskip-4mu/}}  % 平行符号
\renewcommand{\proofname}{\indent\bf 证明}  % 自定义证明标题
\renewcommand{\qedsymbol}{\hfill$\square$}  % 自定义证毕符号
\newcommand{\e}{\mathrm{e}}  % 自然底数
\newcommand{\dash}{\,–\,}  % 短折号
\newcommand{\tong}[1]{\overset{#1}{\equiv\joinrel\equiv}}  % 同余等号
\newcommand{\di}[1]{\,\mathrm{d}#1}  % 微元d
\newcommand{\qu}[2]{\displaystyle\left(#1;#2\right)}  % 开区间
% 局部展开 developpements limites
\newcommand{\oveq}[1]{\overset{#1}{=}}   % equal over
\newcommand{\olim}[1]{\mathit{o}\left(#1\right)}  % petit o
\newcommand{\Olim}[1]{\mathcal{O}\left(#1\right)}  % grand O
\newcommand{\Tlim}[1]{\mathcal{\Theta}\left(#1\right)}  % grand theta
\newcommand{\eqlim}[1]{\overset{#1}{\sim}}  % equivalence
\newcommand{\vect}[1]{\left\langle #1 \right\rangle}  % 生成空间 generated space

\newcommand{\arccot}{\operatorname{arccot}}  % 反余弦函数
\newcommand{\dlim}[1]{^{\color{gray}\prime}#1}  % 数字分隔符
\newcommand{\lian}[1]{  % 极限符号
    \underset{#1}{\operatorname{lian}\,}
}
\newcommand{\nji}[2]{\displaystyle\left( #1 \,|\, #2 \right)}  % 内积
\newcommand{\dangle}{  % 角符号
    \mathord{
        \text{  %
            \tikz[baseline] \draw (0.8em,0ex) -- (0.3em, 0ex) -- (.6em, 1.5ex) -- (.8em, 1.5ex) -- (.5em, 0ex) -- cycle;
        }
    }
}
\newcommand{\xangle}{  % 角符号
    \mathord{
        \text{%
        \tikz[baseline] \draw (0.8em,1.5ex) -- (0.3em, 0ex) -- (.64em, 0ex) -- (.8em, .36ex) -- (.42em, .36ex) -- cycle;
        }
    }
}
\newcommand{\bu}{  % 补集符号
    \mathbin{
        \text{
            \tikz[baseline=-0.6ex]{
                \node[draw, fill=black, minimum size=0.8ex, inner sep=0pt, rectangle] (bu) {};
                \node[draw=none, fill=white, minimum size=0.6ex, inner sep=0pt, circle] at (bu.center) {};
            }
        }
    }
}
\newcommand{\rectbx}{  % 长方形符号
    \mathord{
        \text{%
            \tikz[baseline] \draw (0,.1ex) -- (.4em,.1ex) -- (.4em,1.5ex) -- (0em,1.5ex) -- cycle;
        }
    }
}
\newcommand{\tr}{  % 矩阵转置符号 A^{\tr} 
    \mathord{
        \begin{tikzpicture}[baseline=-0.2em, line width=0.3pt]
        \draw (-0.15em, 0.15em) -- (0.06em, -0.06em);
        \draw (45:0.15em) arc[start angle=45, end angle=225, radius=0.15em];
    \end{tikzpicture}
    }
}
\newcommand{\arcangle}{\mathord{\mathpalette\doarcangle\relax}}  % 带弧的角度符号 - 交角
\newcommand{\doarcangle}[2]{  % 
    \hbox{%
        \sbox0{$#1B$}%
        \sbox2{$#1<$}%
        \raisebox{\dimexpr\dp0+(\ht0-\ht2)/2}{%
            $#1<\mspace{-9mu}\mathrel{)}\mspace{2mu}$%
        }%
    }%
}
\newcommand{\parasbx}{  % 平行四边形符号
    \mathord{
        \text{%
            \tikz[baseline] \draw (0,.1ex) -- (.8em,.1ex) -- (1em,1.6ex) -- (.2em,1.6ex) -- cycle;
        }
    }
}
\usetikzlibrary{calc,topaths}
\newcommand{\widearc}[1]{  % 可伸缩圆弧符号
    \tikz[baseline=(wideArcAnchor.base)]{
        \node[inner sep=0] (wideArcAnchor) {$#1$}; 
        \coordinate (wideArcAnchorA) at ($(wideArcAnchor.north west) + (0.15em,0.1em)$);
        \coordinate (wideArcAnchorB) at ($(wideArcAnchor.north east) + (0.0em,0.1em)$);
        \draw[line width=0.1ex,line cap=round,out=45,in=135] (wideArcAnchorA) to (wideArcAnchorB);
    }
}

% 定义、定理、证明等块环境
\theoremstyle{definition}
\newtheorem{df}{定义}[section] 
\newtheorem*{po}{公理}
\newtheorem{pp}{命题}[section]
\newtheorem{tm}{定理}[section]
\newtheorem{cor}{推论}[pp]
\newtheorem{ex}{例子}[section]
\newtheorem{et}{例题}[section]
\newtheorem*{ex*}{例子}
\newtheorem*{so}{解答}
\theoremstyle{plain}
\newtheorem{sk}{思考}[section]
\newtheorem{xt}{习题}[section]
\renewenvironment{proof}{\paragraph{\textbf{证明:}}}{\hfill$\square$}
% \declaretheorem[name=定义, numberwithin=section, shaded={rulecolor={rgb}{0.1,0.7,0.4},
% rulewidth=2pt, bgcolor={rgb}{0.96,1,0.99}}]{df}
% \declaretheorem[name=定理, numberwithin=section, shaded={rulecolor={rgb}{0.1,0.4,0.7},
% rulewidth=2pt, bgcolor={rgb}{0.96,0.99,1}}]{tm}
% \declaretheorem[name=思考, numberwithin=section, shaded={rulecolor={rgb}{0,0.7,0.7},
% rulewidth=2pt, bgcolor={rgb}{0.98,1,1}}]{sk}
% \declaretheorem[name=习题, numberwithin=section, shaded={rulecolor={rgb}{0.91,0.84,0.42},
% rulewidth=2pt, bgcolor={rgb}{1,0.98,0.93}}]{xt}

\setlength{\intextsep}{2pt}%
\setlength{\columnsep}{2pt}%
% 列举环境
\setlist{label=\textbullet}
% 列举环境行间距
\setenumerate[1]{itemsep=0pt,partopsep=0pt,parsep=0pt,topsep=0pt}
\setitemize[1]{itemsep=0pt,partopsep=0pt,parsep=0pt,topsep=0pt}
\setdescription{itemsep=0pt,partopsep=0pt,parsep=0pt,topsep=0pt}
% 章节间距
\setlength\parskip{8pt}
% 文本框间距
\setlength{\fboxsep}{12pt}
% 章节字体大小
\titleformat{\section}{\zihao{-2}\bfseries}{ \thesection }{16pt}{}

% 封面
\title{\zihao{0} \bfseries 第一册}
\author{\zihao{2} \texttt{大青花鱼}}
% \date{\bfseries\today}
\date{}
% 正文
\begin{document}
\maketitle
\tableofcontents
\newpage

\chapter{什么是物理学}

\section{引言}

你即将开始学习一门新的课程——物理学。研究万物的道理,是为物理学。千百年来,人类探索自己所在的世界是如何运转的。物理学就是诸多探索者给我们留下的知识。

物理学研究我们身边发生的最基本的现象:物体如何运动,为什么有的东西热、有的东西冷,色彩是什么,声音是什么,电和磁又是怎么回事……我们希望从无数看似纷乱无序的现象中总结出规律来,指导我们的生活和生产,让我们的生活更美好、更安全。更进一步的话,我们希望彻底理解我们所在的宇宙,理解自己存在的理由和意义。在长期的探索中,人们不仅总结出了许多规律和知识,更重要的是总结出了一套行之有效的思考方法,一种通过观察、测量、猜想和验证来理解自然现象的探索方式。

你可能会想:既然物理学是要找到世界运转的规律,为什么不一开始就把所有正确的规律告诉我们,然后教我们如何应用这些规律呢?这听起来很合理,但实际上并不容易做到。

原因有三个:

第一,我们至今还没有找到所有的规律。物理学仍在发展,每年科学家们都会有新的发现。

第二,即使是已知的定律,要正确理解它们也需要一步一步地来。很多定律需要借助高深的数学工具和深刻的物理概念才能正确叙述出来,而这些工具和概念又需要借助更多的知识来理解。我们目前掌握的知识,距离正确理解这些定律还比较遥远。

第三,我们至今仍然没有完全了解整个宇宙。人类至今尚未能真正离开地球,到其他的星球自在地调查研究。人类能观测到的宇宙也并非它的全部。即便对于我们能观测到的部分,我们总结出的规律也不足以完美地解释其中发生的所有事情。事实上,物理学家从古至今所作的事情,就是不断总结出能解释更多事情的规律,来替代以往的规律。但我们也无法说,一定能找到可以完美解释一切的方法。实际上,我们知道的一切,都是对宇宙真实的某种近似;我们总结的规律,也只是对宇宙运转真理——如果真有这样的真理的话——的逼近。

因此,我们将采取循序渐进的方式来学习。我们将从一些比较简单、易懂的概念和定律开始。它们来自比较陈旧的“版本”,是古代研究者对自然现象的总结。它们也许并不是最精确最合理的,无法解释所有的问题,但在日常生活中非常好用。随着我们学习不断深入,掌握更多的知识,我们将有能力理解它们的不足,并不断将这些概念和定律更换为更好的“版本”。每一次更新,不是推翻已有的结论,而是增强它们的能力。旧版本能解释的问题,新版本的概念和定律仍然能解释,而且解释得更好。换句话说,我们会不断审视已知的事情,用新的方式看待它,以修正先前对它的谬见。

具体来说,在中学阶段,你将初步理解一套完整的经典理论框架,它代表了19世纪科学家对世界的认识,指导了第一次与第二次工业革命。与此同时,你也会了解到这套框架的界限,并接触现代物理学的基本思想。

这不是说我们一开始学的是“错的”,后来学的是“对的”。我们只是谦虚地承认自己能力的局限。我们可以不知道什么是对的,但绝不把错的当作对的。可以说,迄今所有的物理学理论都是错的。科学的进步就在于不断发现旧理论不适用的地方,并不断提出更好用的新理论,来解释旧理论无法解释的现象。

我们最重要的理念只有一个:实验是一切知识的试金石。实验是检验真理的唯一标准。实验方法是千百年来探索物理的人得到的最珍贵的宝物。我们将要努力理解并掌握实验的方法。

但实验只是一种手段。什么才是知识的源泉呢?那些要检验的知识又从何而来呢?实验为我们提供了种种线索。实验促成了这些知识的产生。但是,要从这些线索中作出最终的判断,还需要人的想象力去猜测蕴藏在所有这些线索后面的答案,找到那简单而优美的解释,然后再用实验来验证我们的猜测究竟对不对。这个想象过程是很艰难的,因此,物理学研究有所分工:理论物理学家进行想象、推演和猜测,但并不做实验;而实验物理学家则进行实验,检验各种理论。

对于中学的学生来说,最重要的不是记住多少公式和结论,而是培养科学的态度和方法。

科学的态度是诚实的:我们承认我们知道的有限,我们愿意根据新的证据修正我们的理解。科学的方法是严谨的:我们重视观察和实验,我们注重正确推理,我们要求理论能够经受检验。每当我们引入一个新的想法,都要问自己:它是否与已知的事实矛盾?它意味着什么新的结果?在什么情况下它可能需要修正?

带着这样的理解,让我们开始认识世界的本质。我们首先要问的问题是:在现代物理学眼中,世界的总体图像是怎样的呢?

\subsection{物理研究中的基本概念}

科学研究的基本步骤和概念:

\begin{enumerate}[label=\textbf{\arabic*.}]
    \item \textbf{观察}:观察是人类理解自然界的基本方式,通过人类的感官(特别是视觉)了解世界,获取自然现象的信息,称为观察。观察是科学研究的起点,它提供原始材料。
    \item \textbf{测量}:测量是更为精确地获取自然现象的信息的手段。通过测量能获得具体的数值(称为数据),为我们做进一步的比较和分析提供依据。观察并测量合称观测。
    \item \textbf{猜想}:基于一定的观察和测量数据,提出的初步的、未经严格检验的想法。猜想通常依靠直觉或经验,需要一定的想象力。它可能指向某种规律,但尚未形成可检验的理论。
    \item \textbf{规律}:人对自然现象的归纳总结,认为繁多的自然现象可以通过一些简单明了的规则解释,这些规则就如同自然界的“法律”。猜想往往以规律的形式表达。
    \item \textbf{假说}:基于猜想和已有的知识,通过思考、推理而构建的,可以检验真伪的明确命题。可以检验真伪是假说的关键要素。
    \item \textbf{公式}:用数学式明确阐述规律,就是公式。假说往往用公式来明确表述规律,以供检验。
    \item \textbf{预测}:假说应当提供对现实的预测,告诉我们在特定条件下,现实将呈现何种结果,如何观测,观测到的数据应该在什么范围。
    \item \textbf{实验}:为检验假说而进行的有控制、可重复的观察过程。假说必须提供如何检验自身真伪的合理方法,而实验就是具体的实现方式。通过以特意设计的方式干预现实,观测结果,我们就可以检验假说的预测是否成真。
    \item \textbf{证伪}:证明某个假说不真。具体来说,就是按照假说提供的预测方法进行实验。如果实验观测结果不在假说预测的范围内,就说假说被证伪了。如果在范围内,说明假说经过了检验。具体细节需要用更多数学知识来描述,但大致如此。
    \item \textbf{理论}:有狭义和广义的含义。广义上来说,理论就是假说,甚至某些合理的猜想也可以称为理论。狭义上来说,理论是经过大量检验、被科学共同体广泛接受的假说,能够解释广泛的现象并做出成功的预测。理论是科学知识的成熟形态,能有效指导现实研究、生产和生活。
    \item \textbf{定律}:有狭义和广义的含义。狭义上来说,经过检验的理论里阐述的规律称为定律。定律往往以简明扼要的公式表述,揭示了自然现象中可重复观察到的稳定关系。定律可以精确地指导研究、生产和生活。广义上来说,也有人会把猜想和假说里提到的规律称为定律。这种说法往往体现了他们对猜想和假说的主观态度,即认为猜想和假说很可能是真的。
    \item \textbf{机制}:机制是导致自然现象发生的内在原因、过程或结构,也称为机理。它解释了现象为什么以某种方式发生,某种规律为何存在。理论往往包含了机制和定律。定律描述关系,但不解释机制。
\end{enumerate}

\section{穷物之理——物理学的前身}

人类认识世界的过程,是从观察身边发生的自然现象开始的。自古以来,各个文明中的人们都注意到日月星辰的运转、季节冷暖的交替、风雨雷电的变化,并尝试理解这些现象背后的规律。这些观察并不仅仅出于好奇,还与人类的生产生活紧密相连。比如,古代中国人通过观测北斗七星的位置变化来判断季节,安排耕作收成的时间;根据地势、水流、风向来修筑水利、建造房屋;积累对金石性质的理解,来冶炼、锻造、烧制出钢铁陶瓷的器物。长期观察总结的实践经验,帮助人们更好地适应自然环境,改善生活条件。

相比于具体的实践经验和技艺,有人更进一步,希望理解万事万物背后的原理。既然人们在各种自然变化中发现了各种规律,那么这些规律是怎么产生的?是谁设置的呢?这疑问自古以来盘旋在人们的心头。

远古时期的中国人,把天地变化、自然奥妙归结为“帝”,认为“帝”能命令天地,降下祸福。通过祭祀与“帝”沟通,可以预知未来。“帝”后来演变成“天”、“老天爷”的概念。荀子曰:“天行有常,不为尧存,不为桀亡”,就认识到了自然规律不依赖人的主观意志的道理。古代的中国人又把自然规律与人类社会联系起来,认为可以从自然规律中推出社会运行所需的道德准则和实践指南。这种“天人合一”的观念使得古代中国人注重观察记录自然现象、发现规律,但也限制了古代中国人对自然规律的解释和想象,阻碍了进一步理解其背后机制的尝试。东汉时期的思想家王充在他的著作《论衡》中,就对当时流行的观念提出了质疑。在解释雷电现象时,他提出“雷者,太阳之激气也”,认为雷电是阴阳二气激烈冲突的结果,就像冶炼时鼓风炉中迸发的火花和声响。然而,王充的思辩未受重视,淹没在历史的浪花中。

地中海沿岸的古文明则有更自由的见解。古希腊林立的城邦中,不少思想家对自然规律的原因给出自己的理解。比如传说中,古希腊的思想家泰勒斯认为水是万物本源;赫拉克利特认为“宇宙是一团永恒的活火”,火不断燃烧、熄灭,是万物生成与毁灭的根源;而恩培多克勒则认为世界由土、水、气、火四种基本元素构成,通过“爱”与“争”的动力形成万物。

古希腊的思想家们不满足于仅仅指出世界的本源,他们试图建立系统的理论来解释自然现象。其中最具影响力的是亚里士多德。亚里士多德把事物和现象归结为“四因”。这里的“因”并不是我们今天理解的“原因”,而是四种理解事物和现象的方法。举例来说,如何理解一把斧头呢?首先,这把斧头是由一块铁和一根木棍构成的,这是它的“材料因”;铁块一边锋利,另一边便于固定在木棍上,这个形式才叫做斧头,是它的“形式因”;一位工匠捶打铁块,将它磨利,按上木柄,造就这块斧头,这是它的“动力因”;最后,这把斧头是某个农夫想砍材才托工匠造的,它用来砍材,这是它的“目的因”。在亚里士多德看来,一切运动的关键在于目的。事物的材料赋予其潜能,而形式规定了它的现实状态,包含了它要实现的目的。运动把物体潜在的形式转为现实,从而达成目的,这就是它的动力。亚里士多德认为,每个物体都有其“自然位置”:土和水这类重元素的自然位置在地心,而气和火这类轻元素的自然位置在天空。因此,一块石头“自然”落下,而火焰“自然”上升,才能实现其目的。亚里士多德把运动分为自然运动和受迫运动。自然运动是事物内在潜能的实现,而受迫运动违背其内在目的,实现外加者的目的。亚里士多德的理论依赖一些不言自明的道理。比如,他认为“越重的物体下落越快”;又如,他认为最完美的形体是球体,因此天体都应当是球体。亚里士多德提出:使物体离开自然位置的运动,必然有外力来维持。一辆马车需要马不断用力拉动才能前进,一旦停止拉车,马车就会停下来。这观点很符合日常经验,似乎不言自明。

另一方面,古代人对自然的认识受到一神教的影响。公元前后,天主教兴起,并逐渐取得了统治地位,罗马帝国将天主教奉为国教后,天主教廷在上千年间成为欧洲的实际统治者。天主教用神创论解释世界起源和本质,垄断了对自然规律的解释。比如雷电显示了神的意志,日月星辰运转是神的安排,等等。13世纪,天主教廷引入亚里士多德的学说,把他的“目的论”和天主教教义整合,形成一套严密的解释体系。这套体系用神作为宇宙的最初原因和最终目的,给出了对自然的权威阐释。对自然的研究探索被框定在神学的体系内。

但是,随着生产力的发展,人类观察自然、研究自然的能力不断提高,思维水平也不断提高。近东地区,东正教和伊斯兰教的统治者因治理庞大疆域的需求,为学者提供了更自由的研究环境。从东方传入的技术给研究者提供了更好的观测工具。学者们需要更精确地描述自己的经验和想法,几何学、代数学、三角学蓬勃发展。比如,10世纪的伊斯兰学者伊本·海什木首次提出了光线从物体射入眼睛的理论,并阐述了光线的传播原理。此外,复杂机械的出现,让人类得以从另一个角度看待自然规律,即自然规律不需要主观意识来推动,可以根据固有的机制自行运作。另一个让人民对权威产生动摇的原因是肆虐欧洲的黑死病。教会的命令无法阻止传染,教士的祈祷无法避免死亡。人民发现,真实的世界不以主观的意志为转移,理性主义思想开始在人们心头萌芽。

1453年,随着君士坦丁堡陷落,大量文化作品流入欧洲。天主教廷内追求自由思考的人文主义者如饥似渴地吸收新来的知识,将其翻译并介绍给更多人。1492年,哥伦布发现了新大陆。不久后,远航的舰队完成了环球旅行。人们对世界的认知发生了彻底的改变。远洋航行带来的新物种、新地理知识,不断挑战《圣经》的说法,迫使学者承认:自然之书的解读权,不在教廷,而在观察者手中。

16世纪的宗教改革进一步瓦解了知识的单一权威。新教运动宣称,人人都能直接阅读圣经,不再依赖教廷的解释。它背后的思想是:若人人能自主解读《圣经》,为何不能自主解读神的另一个作品——自然呢?新教徒把理解自然、解释自然看作与神沟通的方式,把注意力从对宇宙本质的抽象思考转到理解宇宙如何运行。于是,对航行、军事和工程建筑技术的需求,就顺利转为合乎教义的工作。他们希望把各种自然规律中括为几条广泛适用的真理,再用这些真理构思新的技术,设计新工具。而验证真理的方法就是直面现实,一切以现实为准。神已经将真理摆在自然中,只需要用恰当的手段将其揭示出来。这种把自然物理和神灵意志、道德伦理分离的思想,带领人类进入了一个崭新的时代。


\end{document}